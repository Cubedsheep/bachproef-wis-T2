\subsection{Magnetohydrodynamic shock wave}
In the magnetohydrodynamic case the same initial conditions where used as for the Hydrodynamicone.
A uniform magnetic field in the $x$-direction was added. 
To see the effect of the magnetic field, the simulations were done for the following values for the plasma-$\beta$: $\beta \in \{0.1, 0.5,1,10\}$.
This is the dimensionless $\beta$ as given by \cref{eq:plasma-beta}.
In \cref{fig:MHD-blasts} a snapshot of the blast wave at $t=7.5$ is plotted for these values.

The simulations for the MHD blastwave used a larger domain, 12 by 12 in code units.
This allows us to study the waves at later times when the non-linear effects are smaller because of the lower intensity of the wave.
These results will then be used to compare to the linear MHD theory from \cref{sec:MHD-waves}.
The results of the simulation at smaller times are used to compare the behaviour of a MHD wave to the HD wave.

\begin{figure}[H]
	%\hspace{-1cm}
	\centering
	\includegraphics[width=\linewidth]{images/MHD-blasts.pdf}
	\caption{Plots of the pressure of an MHD blastwave with the same intial conditions as in \cref{fig:HD-blast-short}: high pressure difference for the top row, lower for the bottom row.
	The black circle represents the region of higher pressure in the initial condition.}
	\label{fig:MHD-blasts}
\end{figure}

In the plots with $\beta \in \{0.5,1\}$ there is a clear distinction between a fast shock wave spreading in all directions, and two small waves following the magnetic field.
In the plot with $\beta=0.1$ this fast wave is barely visible nearing the edge of the domain, but the two small waves stand out from the uniform background.
for $\beta=10$ the fast wave is almost perfectly spherical and very visible, while the two smaller waves are very faint.
We observe that if the magnetic field gets stronger, the fast wave becomes faster and most of the energy gets concentrated in the slow-mode, while the fast mode gets less energetyic.
To go a bit more in-depth we plot the diagram depiciting the group speeds over the simulation data and see how well they match. 
This can be seen in \cref{fig:MHD-group1} and \cref{fig:MHD-group2}, where the pressure profiles of the waves are plotted at $t=1.25$.
The white dashed lines show the group speed of the fast mode, the grey solid lines the group speed of the slow mode. 
The black circle is again the region with higher pressure in the initial condition.
The group speed diagram was scaled such that the curve for the fast mode would match with the fast wavemode.

Since the initial condition is not a delta function but a circle with finite radius, we cannot expect a perfect match with the theoretical group speed for linear waves.
Instead, a linear wave will be smeared out around the curves for the group speed, as if they were painted with a large brush.
Furthermore when the pressure difference is higher, the non-linear effects will become stronger leading to deviations of the linear theory.

In figure \cref{fig:MHD-group1} the two senarios with a strong magnetic field are shown (low $\beta$).
The effects of the magnetic field are again very clear, removing the isotropy by deforming the fast magnetoacooustic wavefront, and most strikingly introducing the slow-mode waves following the magnetic field.
Because the slow magnetoaccoustic wave mode only travels along the field lines its energy stays concentrated, This is the main cause of the large difference in amplitude between the slow and fast mode.
When the pressure difference is high the nonlinear effects stay important. 
This causes the slow mode waves in the top row in \cref{fig:MHD-group1} to travel significantly faster than the speed of the linear slow wavemode.
In the bottom row, where the pressure difference is a lot smaller, the results match strikingly well to the linear theory.

We also see again that the wave is faster with a higher pressure difference, like with the HD shockwaves. 
Due to the difficulty of accurately detecting the waves in the simulation data, either because they are realy faint with small or large $\beta$, or there is a lot of interference between the modes when $\beta \sim 1$, no accurate calculations of the wave speed from the simulation data could be made.

When the magnetic field becomes small, a quick comparison shows that the wave speed goes to the wave speed of an HD-wave as expected.
This can be seen by comparing the rightmost graphs in \cref{fig:MHD-group2} with the rightmost graphs in \cref{fig:HD-blast-long}

\begin{figure}[H]
	%\hspace{-1cm}
	\centering
	\includegraphics[width=\linewidth]{images/group-speed-pressure1.pdf}
	\caption{Plots of the pressure wave after $1.25$ time units for different values of $\beta$. 
	The white dotted line represents the theoretical position of the fast mode magneto-accoustic wave. T
	The grey solid lines are the parts of the group speed diagram corresponding to the slow-mode magnetoaccoustic wave.
	The black circle in the center is the region where the pressure is higher in the initial condition.
	The top row are plots for simulations where $p_{in}=5$ at $t=0$, the bottom row simulations with $p_{in}=1.05$.}
	\label{fig:MHD-group1}
\end{figure}

\begin{figure}[H]
	%\hspace{-1cm}
	\centering
	\includegraphics[width=\linewidth]{images/group-speed-pressure2.pdf}
	\caption{Same as \cref{fig:MHD-group1} but for different $\beta$ values.}
	\label{fig:MHD-group2}
\end{figure}


