

While the main focus of this bachelor project is the numerical modeling of waves in the solar corona, some theoretical background is important to frame the results of our simulations.
Furthermore, this knowledge gives some insight in the assumptions that are made in deriving the magnetohydrodynamic (MHD) equations and when they are valid.

\subsection*{Hydrodynamics}

\subsection{Hydrodynamic fluid equations}
The theory in this section is adapted from \cite{notes-fluid-dynamics}. For the first task a non-viscous Newtonian fluid is considered. Heat conduction and dissipation is neglected as well.
This type of fluid obeys the Euler equations for conservation of mass, momentum and internal energy:

{\centering
\noindent \fbox{\parbox{0.5\linewidth}{
\begin{equation}
	\label{eq:ideal-fluid}
\begin{split}
	\frac{d\rho}{dt} + \rho \nabla \cdot \vec{v} &= 0\\
	\rho \frac{d\vec{v}}{dt} &= -\nabla p + \vec{F}\\
	\frac{dp}{dt} - \frac{\gamma p}{\rho} \frac{d\rho}{dt} &= 0
\end{split}
\end{equation}
}}\par}


These are the Euler equations in Lagrangian form, with time derivatives following the fluid, hence the total derivatives with respect to time.
PLUTO does the fluid simulation using a static grid, so we need the equations in Eulerian form with partial derivatives with respect to time.
This change of derivatives can be carried out using the following relation, found in \cite{notes-fluid-dynamics} in section 2.4:
\begin{equation}
	\frac{df}{dt} = \frac{\partial f}{\partial t} + (\vec{v} \cdot \nabla) f
	\label{eq:relation-total-partial}
\end{equation}
where $f(x,y,z,t)$ is a function that describes a property of the fluid. The equations can also be rederived using an Eulerian view. In any case the result is the same:

{\centering 
\noindent \fbox{\parbox{0.7\linewidth}{
\begin{align*}
	\frac{\partial \rho}{\partial t} + \nabla \cdot (\rho \vec{v}) &= 0\\
	\frac{\partial}{\partial t} (\rho \vec{v}) &= \nabla \cdot (-p - \rho \vec{v}\vec{v}) + \vec{F}\\
\frac{\partial}{\partial t} \left( \rho \left( \frac{\vec{v}^2}{2} + \mathcal{U} \right)  \right)  &= \vec{F} \cdot \vec{v} - \nabla \cdot \left( \rho \left( \frac{\vec{v}^2}{2}+\mathcal{U}\right)\vec{v} + p\vec{v})  \right) 
\end{align*}}}
\par}

Next introduce the variable $\vec{m}=\rho \vec{v}$, the momentum density. The energy density $\mathcal{U}$ can be split in the thermal energy $\rho e$ and gravitational potential energy $\rho\Phi$. Let $E_t = e\rho + \frac{\vec{v}^2}{2}$ The only external force is $\vec{F} = \rho\vec{g}$
carrying out these substitutions leads to the equations in section 6 in the PLUTO manual \cite{pluto-manual}: \unsure{is the expression for $\vec{F}$ correct? Looks to be different in equation 2 and 3}

{\centering 
\noindent \fbox{\parbox{.7\linewidth}{
\begin{equation}
	\label{eq:ideal-fluid-pluto}
\begin{split}
	\frac{\partial \rho}{\partial t} + \nabla \cdot \vec{m} &= 0\\
	\frac{\partial \vec{m}}{\partial t} + \nabla \cdot \left( \vec{m}\vec{v} + p \right) &= -\rho\nabla\Phi + \rho \vec{g}\\
	\frac{\partial}{\partial t} \left( E_t+\rho\Phi \right)  + \nabla \cdot \left( \left( E_t + p + \rho\Phi \right) \vec{v} \right) &= \vec{m}\cdot\vec{g}
\end{split}
\end{equation}}}
\par}

Together with an equation of state $\rho e = \rho e(p, \rho)$, which gives the thermal energy as a function of $p$ and $\rho$.
In the remainder of this paper a calorically ideal gas is assumed. This is a gas for which the adiabatic constant $\gamma$ obeys:
\begin{equation}
	\gamma = \frac{f+2}{f}
	\label{eq:adiabatic constant}
\end{equation}
where $f$ is the number of degrees of freedom. the previous relation can be rewritten as
\begin{equation*}
	f = \frac{2}{\gamma-1}
\end{equation*}
And by substituting this equation in the equation that expresses the energy as a function of degrees of freedom the closure relation $\rho e= \rho e(\rho, p)$ is found:
\improvement{Add a reference for this energy equation}
\begin{equation*}
	E_t = \rho e = \frac{f}{2}nk_BT = \frac{p}{\gamma-1}
\end{equation*}
\todo[inline]{short discussion of assumptions made (no viscosity and heat conduction, callorically ideal gas)}
\todo[inline]{afleiding golven, groepssnelheid}
\subsection{Hydrodynamic linear waves}
We start again from the ideal fluid equations as given in \autoref{eq:ideal-fluid} and linearize them.
For this we rewrite the quantities $\rho$ and $p$ as a background density $\rho_0$ and pressure $p_0$ with slight deviations $\rho'$, $p'$.
Furthermore it is assumed that there are no external forces acting on the fluid. The Linearized equations are:

{\centering 
\noindent \fbox{\parbox{0.5\linewidth}{
\begin{equation}
	\label{eq:ideal-fluid-linear}
	\begin{split}
		\frac{\partial\rho'}{\partial t} + \rho_0\nabla\cdot \vec{v} &= 0\\
		\rho_0 \frac{\partial \vec{v}}{\partial t} &= -\nabla p'\\
		\frac{\partial p'}{\partial t} &= \frac{\gamma p_0}{\rho_0} \frac{\partial\rho'}{\partial t}
	\end{split}
\end{equation}
}}
\par}
By acting with $\nabla$ on the second equation and using the first to substitute $\rho_0 \nabla \cdot \vec{v}$ we find the following relation:
\begin{equation*}
	\frac{\partial^2 \rho'}{\partial t^2} = -\nabla^2 p'
\end{equation*}
Acting with $ \frac{\partial}{\partial t}$ on the last equation and substituting the previous expression yields
\begin{equation*}
	\frac{\partial^2 p'}{\partial t^2} + \frac{\gamma p_0}{\rho_0} \nabla^2 p' = 0
\end{equation*}
which is the wave equation with $v_s = \sqrt{ \frac{\gamma p_0}{\rho_0} }$ the phase speed of the wave.
Similar expressions are found for the other variables. \improvement{Referentie?}
this wave speed can be found by substituting a plan wave of the form $p' = A \exp \left( i(\omega t - \vec{k}\cdot\vec{x}) \right) $. Substituting this expression in the wave equation for $p'$ leads to the dispersion relation:
\begin{equation}
	\omega^2 = k^2v_s^2.
\end{equation}
The phase velocity is given by
\begin{equation}
	v_{ph} = \frac{\partial \omega}{\partial k} = v_s
\end{equation}
\improvement{reference for this relation for the phase speed?}
from which we conclude that these waves are non-dispersive.

\subsection{Hydrodynamic shocks}

Now we shall reconsider one of the least convincing assumptions made for the derivations of the fluid equations: that of perfectly continuous background variables. In reality, we might encounter very sudden changes in the scalar variable density $\rho$ and vectorial variable velocity $\vec{v}$. To have the theory of ideal fluids take this into account, we can introduce these jumps in the variables as mathematical discontinuities. This discontinuity is appropriately called a 'shock'. We are interested in how this shock moves through the fluid. The derivation of its motion is quite straight forward.\\
\\
Start from the continuity equation in its Eulerian form in 1D

\begin{equation}
\label{eq:continuity-Euler}
\frac{\partial \rho}{\partial t} + \frac{\partial(\rho v)}{\partial x}
\end{equation}

Of course, this equation assumes that $\rho$ and $\rho v$ are continuous variables with continuous partial derivatives. Rewrite the equation so that over a distance $\Delta x$ and a duration $\Delta t$ the variables $\rho$ and $\rho v$ experience a change $\Delta\rho$ and $\Delta \rho v$. This gives the much less elegant version 

$$ \frac{\Delta \rho}{\Delta t} + \frac{\Delta(\rho v)}{\Delta x} = 0 \ . $$

If this were the perfectly continuous case we would now let $\Delta x, \Delta t \to 0$, resulting in \autoref{eq:continuity-Euler}. However, we might also say that the transition is not smooth and that for $\Delta x, \Delta t \to 0$ the jump remains: $ \Delta \rho, \Delta \rho v \to \Delta \rho, \Delta \rho v $. Rewrite the equations to see what this means:

$$ \frac{\Delta x}{\Delta t} + \frac{\Delta(\rho v)}{\Delta \rho} = 0 \ . $$

Then for $\Delta x, \Delta t \to 0$ we get 

\begin{equation}
\label{eq:HD-shock-condition}
\frac{\partial x}{\partial t} + \frac{\Delta(\rho v)}{\Delta \rho} = -V_S + \frac{\Delta(\rho v)}{\Delta \rho} = 0
\end{equation}

where $V_S$ is the shock speed. This relation is the hydrodynamic shock condition. To generalize it beyond 1D, it suffices to take $\vec{v} \cdot \vec{n}$ instead of $v$ where $\vec{n}$ is the unit normal vector on the shock wave front pointing towards the region with lower pressure. It looks as follows

\begin{equation}
\label{eq:HD-3D-shock-condition}
- V_S + \frac{\Delta(\rho \ \vec{v} \cdot \vec{n})}{\Delta \rho} = 0 \ .
\end{equation}

The minus sign in front of $V_S$ is merely a matter of orientation. In \autoref{eq:HD-shock-condition} the orientation is along the positive x-axis. In \autoref{eq:HD-3D-shock-condition} it is along the unit vector $\vec{n}$.

\subsection{Magnetohydrodynamic fluid equations}
There are two approaches commonly taken in the literature to derive the MHD equations. They are either derived from kinetic gass theory, or postulated with added justification of why they can accuratly describe plasmas.

A plasma is an ionised gas consisting of postive and negative ions. In the case of the corona of the sun this is mainly ionised hydrogen.
Therefore the negative ions are free electrons and the positive ions protons, which are a lot heavier than electrons.
When the characteristic timescales $\tau_e$ and $\tau_i$ between two collisions of electrons, respectively ions, is much shorter than characteristic timescales $\tau_f$ at which macroscopic variables change we can use a fluid description. 
At these timescales the individual interactions of individual particles are not relevant anymore.

The plasma can then be described as two different fluids, commonly refered to as the \emph{two-fluid theory}.
The electron gas is one fluid and the proton gas the other. 
The next assumption that is made, is that the relaxation time $\tau_T$ until the electron fluid and ion fluid are in thermal equilibrium after a slight disturbance is also a lot smaller than $\tau_f$.
Finally, we assume that the fluid has no net charge. Not globally, but also not locally. 
This means that in every large enough volume, for every ion with charge $Z$, there are also about $Z$ electrons in this volume.
When all this applies, the variables describing the different fluids can be averaged or added together, to describe the plasma as one fluid.\improvement{add reference to cursus Poedts and course notes arxiv}

The MHD equations can then be found by adding the maxwell equations to the HD equations. Because the HD equations are invariant under Galilean transformations. 
However the Maxwell equations are invariant under Lorentz transformations, so we cannot simply add them to the HD equations and expect a consistent picture. 
Understanding the averaging process is important for understanding what the plasma variables actually represent.
Denote with $n_\alpha$ the number density of a certain type of particle, $m_\alpha$ the mass, $\vec{u}_\alpha$ the velocity of a fluid element and with $p_\alpha$ the pressure of the gas of these particles. 
Let the subscript $e$ denote variables concerning the electrongas and $i$ variables describing the iongas.
The variables describing the plasma are the following linear combinations of variables describing the electron and ion gas:

{\centering 
\noindent \fbox{\parbox{.5\linewidth}{
\begin{equation}
	\label{eq:plasma-variables}
	\begin{split}
		\rho &= n_em_e+n_im_i\\
		\vec{v} &= (n_em_e\vec{u}_e + n_im_i\vec{u}_i)/\rho\\
		\vec{j} &= -e(n_e\vec{u}_e-Zn_i\vec{u}_i)\\
		p &= p_e+p_i
	\end{split}
\end{equation}
}}
\par}

Where $e$ is the charge of an electron and $Z$ the charge of an ion as a multiple of the electron charge. The first equation is the \emph{total mass density}, the second the \emph{center of mass velocity}, the third the \emph{current density} and the last one describes the \emph{total pressure}.

For a consistent Newtonian theory of MHD, the displacement current $\epsilon_0 \frac{\partial \vec{E}}{\partial t}$ is neglected. \improvement{reference to lecture notes arXiv}

Finally, the viscosity and heat flow are neglected like in the HD case. Furthermore, for the ideal MHD case the resistivity of the fluid is neglected. The extra equations we need are then:
\begin{align*}
	\frac{\partial \vec{B}}{\partial t} &= - \nabla \times \vec{E}\\
	\nabla \cdot \vec{B} &= 0\\
	\nabla \times \vec{B} &= \mu_0 \vec{j}
\end{align*}
We do not need an equation relating the charge distribution to the electric field in the first equation since we assumed the fluid is locally neutral.
Furthermore the displacement term in the third equation was neglected.

Adding everything together such as in [REFERENCE TO ONE OF THE COURSES]\improvement{reference to course notes poedts} yields the ideal MHD equations:

{\centering 
\noindent \fbox{\parbox{.85\linewidth}{
\begin{equation}
	\label{eq:ideal-MHD}
	\begin{split}
		\frac{\partial\rho}{\partial t} + \nabla\cdot (\rho\vec{v}) &= 0\\
		\rho \left( \frac{\partial\vec{v}}{\partial t} + \vec{v}\cdot\nabla\vec{v} \right) + \nabla p - \vec{j} \times \vec{B} &= 0\\
		\frac{\partial p}{\partial t} + \vec{v}\cdot\nabla p + \gamma p\nabla\cdot\vec{v} &= 0\\
		\frac{\partial \vec{B}}{\partial t} - \nabla\times (\vec{v}\times\vec{B}) &= 0
	\end{split}
\end{equation}
}}
\par}

Where $$\vec{j} = \frac{\nabla\times \vec{B}}{\mu_0}. $$ We need one additional equation which the inital condition has to satisfy:
\begin{equation*}
	\nabla\cdot\vec{B}=0
\end{equation*}
which expresses that there are no magnetic monopoles. By acting with $\nabla\cdot$ on the fourth equation in \autoref{eq:ideal-MHD} we see that if the initial equation satisfies $\nabla\cdot\vec{B}=0$, it is automatically satisfied for all later times:
\begin{equation*}
	\frac{\partial (\nabla\cdot\vec{B})}{\partial t} = 0
\end{equation*}

The equations used by PLUTO in the ideal case have a slightly different form: \improvement{reference to user guide}

{\centering 
\noindent \fbox{\parbox{.85\linewidth}{
\begin{equation}
	\label{eq:ideal-MHD-PLUTO}
	\begin{split}
		\frac{\partial \rho}{\partial t} + \nabla \cdot (\vec{m}) &= 0\\
		\frac{\partial\vec{m}}{\partial t} + \nabla \cdot \left[ \vec{m}\vec{v} - \vec{B}\vec{B}+I \left( p+ \frac{\vec{B}^2}{2} \right)  \right] ^T &= -\rho\nabla\Phi+\rho \vec{g}\\
		\frac{\partial\vec{B}}{\partial t} + \nabla \times (c\vec{E}) &= 0\\
		\frac{\partial (E_t+\rho\Phi)}{\partial t} + \nabla \cdot \left[ \left( E_T+p_t+\rho\Phi \right)\vec{v} - \vec{B}(\vec{v}\cdot\vec{B})  \right] &= \vec{m}\cdot\vec{g}
	\end{split}
\end{equation}
}}
\par}

where, as with the HD equations, $\vec{m}=\rho\vec{v}$ and $E_t$ is again the total energy density, this time with an extra term for the magnetic field:
\begin{equation*}
	E_t = \rho e + \frac{\rho \vec{v}^2 + \vec{B}^2}{2}
\end{equation*}
$c\vec{E}$ is given by:
\begin{equation*}
	c\vec{E} = -\vec{v}\times \vec{B}
\end{equation*}
note that the equations do not formally depend on the speed of light, but it is kept in the equations for consistency with the relativistic case.
\subsection{Magnetohydrodynamic waves}


For our discussion of magnetohydrodynamic we rewrite the basic MHD equations into a form which is easier to linearalize. We also assume that the plasma is completely homogeneous and in equilibrium. In practice this means that we shall remove the cross products in the original equations as follows 

{\centering 
\noindent{
\begin{equation}
	\begin{split}
		- \vec{J} \times \vec{B} &= -(\nabla  \times \vec{B}) \times \vec{B} = (\nabla \vec{B}) \cdot \vec{B} - \vec{B} \cdot \nabla \vec{B}\\
		\nabla \times \vec{E} &= -\nabla \times (\vec{v} \times \vec{B}) =  \vec{B} \nabla \cdot \vec{v} + \vec{v} \cdot \nabla \vec{B} - \vec{B} \cdot \nabla \vec{v}
	\end{split}
\end{equation}
}
\par}


The MHD equations now become

{\centering 
\noindent \fbox{\parbox{.85\linewidth}{
\begin{equation}
	\label{eq:ideal-MHD-withoutcross}
	\begin{split}
		\frac{\partial\rho}{\partial t} + \nabla\cdot (\rho\vec{v}) &= 0\\
		\rho \frac{\partial \vec{v}}{\partial t} + \rho \vec{v} \cdot \nabla \vec{v} + (\gamma - 1)\nabla (\rho e) + (\nabla \vec{B})\cdot \vec{B} - \vec{B} \cdot \nabla \vec{B} &= 0\\
		\frac{\partial e}{\partial t} + \vec{v} \cdot \nabla e + (\gamma -1)e \nabla \cdot \vec{v} &= 0\\
		\frac{\partial \vec{B}}{\partial t} + \vec{v} \cdot \nabla \vec{B} + \vec{B} \nabla \cdot \vec{v} - \vec{B} \cdot \nabla \vec{v} &= 0
	\end{split}
\end{equation}
}}
\par}

Of course the condition that $ \nabla \cdot \vec{B} = 0  $ remains. In this form the equations are much easier to linearalize:

{\centering 
\noindent \fbox{\parbox{.85\linewidth}{
\begin{equation}
	\label{eq:ideal-MHD-linear}
	\begin{split}
		\frac{\partial\rho_1}{\partial t} + \nabla\cdot (\rho_0\vec{v_1}) &= 0\\
		\rho_0 \frac{\partial \vec{v_1}}{\partial t} + (\gamma - 1)\nabla (\rho_1 e_0) +\nabla (\rho_0 e_1) + (\nabla \vec{B_1})\cdot \vec{B_0} - \vec{B_0} \cdot \nabla \vec{B_1} &= 0\\
		\frac{\partial e_1}{\partial t} + (\gamma -1)e_0 \nabla \cdot \vec{v_1} &= 0\\
		\frac{\partial \vec{B_1}}{\partial t} + \vec{B_0} \nabla \cdot \vec{v_1} - \vec{B_0} \cdot \nabla \vec{v_1} &= 0
	\end{split}
\end{equation}
}}
\par}

The second one of these equations \autoref{eq:ideal-MHD-linear} is the linearalized momentum equation. We shall work from this one as it lends itself the most for our discussion of ideal MHD waves. This is because it directly describes flow velocity. Plugging the other three into this equation gives us the essential equation for ideal MHD waves:

\begin{equation}
	\label{eq:ideal-wave}		
		\frac{\partial^2 \vec{v_1}}{\partial^2 t} = \bigg( (\textbf{b} \nabla)^2 \textbf{I} + (b^2 + c^2)\nabla \nabla - \textbf{b} \cdot \nabla (\nabla \textbf{b} + \textbf{b} \nabla) \bigg) \cdot \vec{v_1}
\end{equation}

where $c = \sqrt{\cfrac{\gamma p_0}{\rho_0}}$ and $\textbf{b} = \cfrac{\textbf{B}_0}{\sqrt{\rho_0}}$ . We introduce this constants $c$ and $\textbf{b}$ as they will be the wave velocities of the solutions of the wave equation \autoref{eq:ideal-wave}. The constant $c$ is the acoustic speed known from regular hydrodynamics. The constant $\textbf{b}$ is know as the \textit{Alfvén } velocity and it is a vector in the same direction as the background magnetic field $\textbf{B}_0$.\\
\\
Notice that if we set $\textbf{b} = 0$ \autoref{eq:ideal-wave} becomes

$$ \frac{\partial^2 \vec{v_1}}{\partial^2 t} = c^2 \nabla^2 \vec{v_1} $$

which is exactly what we would expect as this is wave equation in the normal hydrodynamic case. This is an important sanity check for our method.\\
\\

We shall be looking for sinusoidal wave solutions. For now we shall also limit the discussion the waves in the velocity vector field as the waves in the scalar pressure and density fields and the magnetic vector field can easily be expressed in terms of the velocity field using the equations \autoref{eq:ideal-MHD-linear}. The solutions we look for are of the form

$$ \vec{v_1} = \bar{\textbf{v}} \exp(i(\omega t - \vec{k} \cdot \vec{x})) \ .$$ 

Under the constrain of having to provide sinusoidal wave solutions \autoref{eq:ideal-wave} becomes

\begin{equation}
\label{eq:plane-wave-equation}
\bigg( \big( \omega^2 - (\vec{k}\cdot \textbf{b})^2 \big)\textbf{I} - (b^2 + c^2)\vec{k}\vec{k} + \vec{k} \cdot \textbf{b}(\vec{k}\textbf{b} + \textbf{b}\vec{k}) \bigg) \cdot \bar{\textbf{v}} = 0 \ .
\end{equation} 

In order to solve this we need that 

$$ \det \bigg( \big( \omega^2 - (\vec{k}\cdot \textbf{b})^2 \big)\textbf{I} - (b^2 + c^2)\vec{k}\vec{k} + \vec{k} \cdot \textbf{b}(\vec{k}\textbf{b} + \textbf{b}\vec{k}) \bigg) = 0 \ .$$

We shall begin by interpreting this formula in 2D and then see what happens when a third component is added. Without any loss of generality we can assume that $\textbf{b} = (b, 0)$ and $\vec{k} = (k_x, k_y)$. Broken down into components we then have 

\begin{equation}
\begin{pmatrix}
 -k_x^2 c^2  &  - k_y k_x c^2 \\
- k_y k_x c^2  &    - k_x^2 (b^2 + c^2) - k_y^2 b^2
\end{pmatrix}
\begin{pmatrix}
\bar{v}_x \\
\bar{v}_y
\end{pmatrix}
 =  -\omega^2
\begin{pmatrix}
\bar{v}_x \\
\bar{v}_y
\end{pmatrix}
\end{equation} 

Then we see that the determinant of the system is 

$$ \omega^4 - \omega^2(k_x^2 b^2 - k_y^2 b^2 - k_y^2 - k_y^2) + b^2 c^2 k_x^2(1 + k_y^2) $$

of which the roots are $ \omega^2 = k^2 \sqrt{b^2 + c^2}(\sqrt{b^2 + c^2} \pm bck_x) $ where $k^2 = k_x^2 + k_y^2$ . These solutions correspond to the so-calles fast (+) and slow (-) magnetosonic waves. One can readily see that they are the result of a quite complicated interplay between the hydrodynamic and magnetic sides of the story. To see this better, it helps to lift the problem to 3D.\\
\\
Take $ \vec{b} = (b,0,0), \vec{k} = (k_x, k_y,0) $. There is no loss of generality. Plugging these vectors into \autoref{eq:plane-wave-equation} we get a slightly different determinant.

$$ (\omega^2 - k_x^2 b^2)\big( \omega^4 - \omega^2(k_x^2 b^2 - k_y^2 b^2 - k_y^2 - k_y^2) + b^2 c^2 k_x^2(1 + k_y^2) \big) \ . $$
 
The roots corresponding to the fast and slow magnetosonic waves are again present (as one would expect). However, there is an extra root $\omega^2 = k_x^2 b^2$. This one is of great interest as it does not contain the same complicated magnetosonic interaction and solely depends on the nature of the magnetic field. The density irregularities only provide the wave's momentum. The restoring force is entirely generated by the tension in the magnetic field.\\
\\
The wave corresponding to $\omega_A^2 = k_x^2 b^2$ is called the \textit{Alfvén} wave. Notice that its direction corresponds to that of the magnetic field, where $\omega_A = k_x b$ lies in the same direction and $-\omega_A$ in the opposite direction. It should be noted that this solution is non-relativistic. As the magnetic field becomes stronger in comparison to the density the Alfén wave becomes a regular electromagnetic wave.\\
\\
Now, the first roots we had - the magnetosonic waves - are combinations of Alvén waves and ordinary sound waves. There are two types because the Alfvén and sound waves can either be in fase or in antifase to one another. In the first wave the region of high pressure will correspond to a high magnetic field density, which causes the resulting wave to be driven forward by both ordinary hydrodynamic pressure and the tension of the concentrated magnetic field lines. In the case of the slow wave these same two forces work against each other, slowing the wave.



\todo[inline]{Derivation of MHD equation, discussion of the integration scheme used}
\todo[inline]{Test informatievaardigheden!!!}

