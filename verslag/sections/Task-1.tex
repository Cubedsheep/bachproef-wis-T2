\subsection{Theory Hydrodynamic linear waves}
The theory in this section is adapted from \cite{notes-fluid-dynamics}. For the first task a non-viscous Newtonian fluid is considered. Heat conduction and dissipation is neglected as well.
This type of fluid obeys the Euler equations for conservation of mass, momentum and internal energy:

{\centering
\noindent \fbox{\parbox{0.5\linewidth}{
\begin{align*}
	\frac{d\rho}{dt} + \rho \nabla \cdot \vec{v} &= 0\\
	\rho \frac{d\vec{v}}{dt} &= -\nabla p + \vec{F}\\
	\frac{dp}{dt} - \frac{\gamma p}{\rho} \frac{d\rho}{dt} &= 0
\end{align*}}}\par}

These are the Euler equations in Lagrangian form, with time derivatives following the fluid, hence the total derivatives with respect to time.
PLUTO does the fluid simulation using a static grid, so we need the equations in Eulerian form with partial derivatives with respect to time.
This change of derivatives can be carried out using the following relation, found in \cite{notes-fluid-dynamics} in section 2.4:
\begin{equation}
	\frac{df}{dt} = \frac{\partial f}{\partial t} + (\vec{v} \cdot \nabla) f
	\label{eq:relation-total-partial}
\end{equation}
where $f(x,y,z,t)$ is a function that describes a property of the fluid. Or by doing the derivation of the equations again from start with an Eulerian view. In any case the result is the same:

{\centering 
\noindent \fbox{\parbox{0.7\linewidth}{
\begin{align*}
	\frac{\partial \rho}{\partial t} + \nabla \cdot (\rho \vec{v}) &= 0\\
	\frac{\partial}{\partial t} (\rho \vec{v}) &= \nabla \cdot (-p - \rho \vec{v}\vec{v}) + \vec{F}\\
\frac{\partial}{\partial t} \left( \rho \left( \frac{\vec{v}^2}{2} + \mathcal{U} \right)  \right)  &= \vec{F} \cdot \vec{v} - \nabla \cdot \left( \rho \left( \frac{\vec{v}^2}{2}+\mathcal{U}\right)\vec{v} + p\vec{v})  \right) 
\end{align*}}}
\par}

Next introduce the variable $\vec{m}=\rho \vec{v}$, the momentum density. The energy density $\mathcal{U}$ can be split in the thermal energy $\rho e$ and gravitational potential energy $\rho\Phi$. Let $E_t = e\rho + \frac{\vec{v}^2}{2}$ The only external force is $\vec{F} = \rho\vec{g}$
carrying out these substitutions leads to the equations in section 6 in the PLUTO manual \cite{pluto-manual}:

{\centering 
\noindent \fbox{\parbox{.7\linewidth}{
\begin{align*}
	\frac{\partial \rho}{\partial t} + \nabla \cdot \vec{m} &= 0\\
	\frac{\partial \vec{m}}{\partial t} + \nabla \cdot \left( \vec{m}\vec{v} + p \right) &= -\rho\nabla\Phi + \rho \vec{g}\\
	\frac{\partial}{\partial t} \left( E_t+\rho\Phi \right)  + \nabla \cdot \left( \left( E_t + p + \rho\Phi \right) \vec{v} \right) &= \vec{m}\cdot\vec{g}
\end{align*}}}
\par}

Together with an equation of state $\rho e = \rho e(p, \rho)$, which gives the thermal energy as a function of $p$ and $\rho$.
In the remainder of this paper a calorically ideal gas is assumed. This is a gas for which the adiabatic constant $\gamma$ obeys:
\begin{equation}
	\gamma = \frac{f+2}{f}
	\label{eq:adiabatic constant}
\end{equation}
where $f$ is the number of degrees of freedom. the previous relation can be rewritten as
\begin{equation*}
	f = \frac{2}{\gamma-1}
\end{equation*}
And by substituting this equation in the equation that expresses the energy as a function of degrees of freedom the closure relation $\rho e= \rho e(\rho, p)$ is found:
\begin{equation*}
	E_t = \rho e = \frac{f}{2}nk_BT = \frac{p}{\gamma-1}
\end{equation*}

%TODO: add references to course notes for the thermodynamic identities used in the last part.
