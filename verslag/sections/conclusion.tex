\section{Conclusion}

\subsection*{Summary}

Before tying a more general close to this report, we provide a brief summery of its content. 
Firstly, we have given an elementary discussion of HD and MHD theory to serve as a framework for the analysis of the simulation results. The conclusions we have drawn are mainly qualitative. In particular, we have made several essential remarks concerning waves and shockwaves, as a basic understanding in this area is required to understand and interpret the simulations. To be more specific, we have given an explicit expression for the shock speed. This is a quantitative result which we have used to check the validity and accuracy of the method outlined in this report.
Secondly, we have constructed detailed simulations of both HD and MHD shockwaves and interpreted the output data. The HD case served mainly as a test case. However, we have seen that the speed of the simulated wave for different initial conditions and the shock speed predicted by HD theory line up almost perfectly. Having established the credibility of the method, we repeated the simulations for a nonzero magnetic field strength. As is to be expected, the difference in the results is quite striking. Nevertheless, we have seen once more near-perfect agreement between the theoretical prediction and simulated actuality, both qualitatively and quantitatively.
Finally, we have investigated the influence of a coronal hole on an MHD wave. This problem was certainly more intricate and subtle than the previous one. Therefore we have examined the interaction between the hole and the wave minutely from both `perspectives'. That is, we have analysed both the effects on the wave as on the coronal hole itself. Furthermore, we have had a similar setup with a coronal plume simulate a wave-plume collision. This made for an interesting comparison. In both cases the structure was slightly moved in the direction of the wave and both times a clear reflection and diffraction of the wave is visible. However, there is a clear difference in the diffraction of the transmitted wave after the encounter. 

Last remarks
The central dogma of present-day physics is that the natural laws must apply everywhere. One could make the case that the theory of magnetohydrodynamics is a perfect example of its success as it seals the chasm between the very largest and smallest scales. Its assumptions are made at the level of elementary particles, discrete charges and molecular degrees of freedom, but its implications reach beyond the scale of stars and even the interstellar medium. Most other theories of physics either lose their validity or applicability at one or both of those extreme scales.
To add to that, it is estimated that most – up to 90 $\%$ \citep{notes-principles-MHD}– of the matter in the universe (excluding dark matter) is in a plasma-like state. It is therefore safe to say that the applicability of MHD theory is near universal. Both in thermonuclear reactor experiments as in the study of asteroseismology, one must take into account the additional layer of complexity as a consequence of an underlying magnetic field. Furthermore, we have shown the effects of allowing such a magnetic field to dominate the plasma and overshadow the other forces involved. 
To end the discussion where we began, the solar corona. A thorough understanding of magnetohydrodynamics and sufficiently precise measurement data may allow us to predict powerful solar seismological activity which can even have damaging effects on our day-to-day lives. Think for example of the 1989 fallout in Quebec due to a solar flare in Earth’s direction. With its extreme conditions and immense dimensions, it is difficult to imagine a more suitable setting for the macroscopic study of plasma. The implications of MHD theory are directly observable via satellite measurements and even during a total solar eclipse. 

Acknowledgements
To close, we would like to extend our exceptional gratitude to our project’s counsellor Mijie Shi for his guidance during out time working on this project. We would also like to thank our fellow students Rune Buckinx and Michaël Maex who had the same subject and were kind enough to share some of their research material with us.
