In this section we study the interaction of a so-called "coronal hole" with an MHD wave as introduced in the paper \cite{coronal-hole} by A.N. Afanasyev and A. N. Zhukov. 
A coronal hole is a region in the solar corona with much lower density and temperature than the surrounding area.
The coronal hole was simulated in 2.5d, assuming that the variables do not depend on $z$. However, the $z$-components of vectors do not need to be zero.
An ideal equation of state is assumed. This gives the following relation between density, temperature and pressure:
\begin{equation}
	nk_BT = p \ .
	\label{eq:ideal-gas}
\end{equation}
The total pressure is obtained by adding the magnetic pressure given by \cref{eq:magnetic-pressure} to the mechanical pressure, resulting in:
\begin{equation*}
	p^T = p + \frac{B^2}{2} \ .
\end{equation*}
From there, the magnitude of the magnetic field is calculated as follows (expressions for code units):
\begin{equation}
	B = \sqrt{2 \left( p^T-p \right) } \ .
	\label{eq:magnetic-magnitude}
\end{equation}
The total pressure inside and outside the coronal hole is kept in equilibrium by changing the magnitude of the magnetic field accordingly.
In contrast to the magnetic field in the MHD blastwave simulation, the magnetic field is now assumed to lie in the $z$-direction.

\subsection{Initial and boundary conditions}

The parameters for the coronal hole are the same as in \cite{coronal-hole}. 
The physical size of the domain is a square with side length $2\times 10^{11}cm$, the grid used for the simulation has $1024$ by $1024$ gridpoints. As a model for the number density and temperature the following was used:
\begin{equation}
	\label{eq:hole-model}
	\begin{split}
		n(r) &= n_{out} - (n_{out}-n_{in})\exp \left( -(r/d)^8 \right)\\
		T(r) &= T_{out} - (T_{out}-T_{in})\exp \left( -(r/d)^8 \right) 
	\end{split}
\end{equation}
where $r$ represents the distance from the center of the hole and $d$ the characteristic size of the hole. 
The parameters $n_{out}, n_{in}, T_{out}$ and $T_{in}$ are respectively the plasma number density outside and inside the hole, the temperature outside and the temperature inside the hole. 
The magnetic field has zero $x$ and $y$ components and the $z$ component is calculated according to \cref{eq:magnetic-magnitude}.
The total pressure is calculated by fixing a value $B_{out}$ for the magnitude of the magnetic field far away from the hole and calculated using \cref{eq:magnetic-magnitude}.
As a counterpart to the coronal hole model, a coronal plume model was considered as well, again based on \cite{coronal-hole}.
Here the density outside the plume is a lot lower than the density inside.
The parameter values for both models can be found in the following table.
In \cref{fig:hole-initial} and \cref{fig:plume-initial} the value of some important quantities at $t=0$ are shown along a section through the center of the hole or plume respectively.

\begin{table}[H]
	\centering
	\begin{tabular}{|l|r|r|}
		\toprule
		Parameter & Coronal hole & Coronal plume\\
		\midrule
		$n_{out}$ & $1.0\times 10^9$ cm$^{-3}$ & $1.0\times 10^8$ cm$^{-3}$\\
		$n_{in}$ & $1.0\times 10^8$ cm$^{-3}$ & $1.0\times 10^9$ cm$^{-3}$\\
		$T_{out}$ & $1.5\times 10^6$ K & $1.5\times 10^6$ K\\
		$T_{in}$ & $1.0\times10^6$ K & $1.0\times10^6$ K\\
		$d$ & $1.5\times10^{10}$ m& $1.0\times10^{10}$ m \\
		$B_{out} $ & $4.0$ G & $3.0$ G\\
		\midrule
		$v_m$ & $50$ km s$^{-1}$ & $125$ km s$^{-1}$ \\
		$s_1$ & $20$ s & $60$ s \\
		$s_2$ & $16$ s& $50$ s\\
		\bottomrule
	\end{tabular}
	\caption{Parameter values taken from \cite{coronal-hole}. The values used for the initial condition of the coronal hole and coronal plume models.}
	\label{tab:parameters-hole}
\end{table}
For the non-linear wave driver, the velocity along $x$ was perturbed at the left boundary. 
To smoothly pass from $0$ to a certain max velocity $v_m$, stay at this velocity for some time and smoothly return to $0$, a combination of two hyperbolic tangents is used in the following formula:
\begin{equation}
	v_x(t) = v_m \tanh \left( \frac{t}{s_1} \right) - \frac{v_m}{2} \left( \tanh \left( \frac{t-T_0}{s_2} \right) +1 \right) 
\end{equation}
The parameters $s_1$ and $s_2$ control the steepness of the transition from $v=0$ to $v=v_m$.
$T_0$ controls the time when the velocity drops back from $v_m$ to $0$. At $t=0$ the velocity starts increasing immediately.
This wave propagates with the speed of a fast magnetosonic wave transverse to the magnetic field (since the magnetic field points along the $z$-direction).
Using \cref{eq:magnetoaccoustic-group-speed}, we find that the group speed of the fast magnetosonic wave transverse to the magnetic field is:
\begin{equation*}
	v_{g+} = \sqrt{v_a^2+v_s^2}
\end{equation*}
From \cref{fig:hole-initial} and \cref{fig:plume-initial}, it is clear that this speed is a lot higher outside the plume than outside the hole (about $2.5$ times).
To obtain a wave of the same physical width in both simulations, we use a shorter wave pulse in the coronal plume model with slightly steeper edges and higher maximum velocity.
The parameters for the waves can again be found in tab. 1.

At the right boundary, an open boundary condition is used. For the top and bottom boundaries a periodic boundary condition is used.

\begin{figure}[H]
	%\hspace{-1cm}
	\centering
	\includegraphics[width=.9\linewidth]{images/sections-initial-condition-hole.pdf}
	\caption{Sections of the initial condition for the \emph{coronal hole} model. 
		\emph{First plot}: plasma beta as given by \cref{eq:plasma-beta}, dimensionless quantity.
		\emph{Second plot}: the full blue line is the plasma number density in $10^8$ cm$^{-3}$.
The red dashed line is the magnetic field in Gauss.
\emph{Third plot}: the full line is the sound speed $v_s$, the dashed red line the Alfvén speed $v_a$ and the dash-dotted black line the fast magnetosonic speed $v_+$.
\emph{Last plot}: speed profile of the wave driver used.
Adapted from \cite{coronal-hole}}
	\label{fig:hole-initial}
\end{figure}

\begin{figure}[H]
	%\hspace{-1cm}
	\centering
	\includegraphics[width=.9\linewidth]{images/sections-initial-condition-plume.pdf}
	\caption{Same as \cref{fig:hole-initial} but for the \emph{coronal plum}e model.}
	\label{fig:plume-initial}
\end{figure}

\newpage
\subsection{Terminology}
Before we delve into the discussion of our simulation results, first some comments on the terminology used.
Our discussion is mainly based on \cref{fig:hole-frames}, \cref{fig:plume-frames}, \cref{fig:hole-sections} and \cref{fig:plume-sections}.

\Cref{fig:hole-frames} and \cref{fig:plume-frames} contain snapshots at 6 different timestamps.
For each timestamps, two plots are made next to each other.
One plot contains heatmap of the complete simulation domain representing the density. Next to it is again a density heatmap, this time zoomed in on the white box in the plot next to it, providing a better view of the hole/plume.
The limits for the colormap are chosen in such a way to highlight the wave fronts, the units of the labels for the colorbar are $10^9$particles cm$^{-3}$.

When we reference a frame, we talk about the two plots with the same timestamp. If we discuss behaviour inside the hole/plume, this concerns the right plot of the pair.
Behaviour outside the structure is about the left plot of the pair.

\Cref{fig:hole-sections} and \cref{fig:plume-sections} contain plots of the density along a cut through the center of the hole or plume along the direction of propagation of the wave.
The top and middle plot show the wave outside the structure. The bottom plot zooms in on the wave inside the structure.
The dashed lines show the position of the wave front at the bottom edge of the simulation domain.

We will often mention the transmitted and reflected waves, as well as the "main" wave. 
The transmitted wave is the wave that travels through the structure when it reappears the structure.
This is for instance the red ring on the right hand side of the hole in the fourth frame in \cref{fig:hole-frames} and the red ring around the plume in the last frame of \cref{fig:plume-frames}.
The reflected wave refers to the wave coming from the left side, top and bottom of the hole/plume right after the main wave has come into contact with the structure.
The main wave is the wave generated at the left boundary.


\begin{figure}[H]
	%\hspace{-1cm}
	\centering
	\includegraphics[width=\linewidth]{images/hole-sections.pdf}
	\caption{Density profiles along a cut, parallel to the direction of propagation of the wave, through the center of the \emph{coronal hole} at different times. The vertical dashed lines show the position of the wave front at the bottom edge of the simulation domain.}
	\label{fig:hole-sections}
\end{figure}

\subsection{Results coronal hole}

In \cref{fig:hole-frames}, a couple of density profiles of the \emph{coronal hole} simulation are plotted.
From \cref{fig:hole-initial} we see that the speed of the wave (the fast magnetosonic speed) is a lot higher inside than outside the hole.
Therefore, it is to be expected that the wave transmitted through the hole will be faster then the wave going around.
This effect is visible on the heatmaps in \cref{fig:hole-frames}. 
In the image at $t=2.97e+3s$, the wave front of the transmitted wave is barely visible as the red border to the right of the coronal hole. It is a significant distance ahead of the main wave front.
The frames at later times show that the main wave slowly catches up with the transmitted wave. This is probably due to non-linear effects in the shock-wave.

\begin{figure}[H]
	%\hspace{-1cm}
	\centering
	\includegraphics[width=\linewidth]{images/hole-frames.pdf}
	\caption{Plots of the density profile of the \emph{coronal hole model} for different times. 
	The plots in the first and third column cover de complete domain of $2e11\times2e11$cm.
	The plots in the second and last column are zoom in on the plume, the units for the axes are $1e11$cm. 
	The domain covered is the white box in the plots to the left.
 The density is measured in $10^{9}$ cm$^{-3}$.
The white circle has the characteristic width $d$ as diameter. 
In the second and third pair of plots, the density range for the right plot is taken a lot more narrow to highlight the wave transmitted through the coronal hole. }
	\label{fig:hole-frames}
\end{figure}


This fast transmission is the most visible in \cref{fig:hole-sections}. 
At $t=2.50e+3 s$, the wave is almost at the center of the hole. Whereas the wave front at the edge of the simulation domain (the green dashed line) is already lagging $100\times10^{8} cm$ behind.
Comparing this to the snapshot at $t=2.55e+3$, shows that the wave front inside the hole, which has travelled about $50\times10^{8}$ cm since the previous snapshot, propagates about $2.5$ times faster than the wave outside the hole, covering only $20\times10^{8}$ cm in the same timespan.
We also remark that the effect of the wave on the hole is a sudden increase in density. This is the wave front seen in the snapshots previously discussed.
When the wave has passed, the density drops back to its original value. However, the position where it reaches its lowest value is shifted along the direction of propagation of the wave.

Two other phenomena that are readily observed in the last three frames are the diffraction of the main wave around the hole and a rarefraction wave reflected from the hole.
At $t=2.97e+3$ and $t=3.42e+03$, we see the wave bending around the hole and interfering with the transmitted wave.
In the last frame, the diffracted wave is visible behind the transmitted wave front, both slightly less energetic than the main wave.
The incoming wave also reflects from the hole.
In the fourth frame a black region just behind the hole is visible. This is the reflected rarefraction wave (wave with lower density, in contrast to a shock wave with higher density), closely followed by a wave front with slightly higher density.
We see that the reflected wave has opposite phase compared to the incoming wave.
Because this wave is almost perfectly circular with the same radius as the transmitted wave, measured from the center of the hole, it is clear that this wave travels at the fast magnetosonic speed as well.

These two effects alter the density of the plasma. The transmitted wave has a lower density compared to the main wave, but higher than the surrounding plasma, whereas the rarefraction wave has a lower density than the surrounding.
Therefore, by altering the density they alter the brightness of the plasma. 
If the hole itself is not directly observable, local dimming of a wave, together with a dimmer wave travelling backwards, may point to a local sudden decrease in density.
Furthermore, by carefully studying the form of these waves and amount of dimming, it might be possible to extract more information about the parameters of the non-uniformity such as size and relative change in density.

Lastly, we note that the hole's position is slightly shifted along de direction of propagation of the wave.
Apart from a slight deformation of the center of the hole, where the density is almost uniform, its shape remains largely unaltered.


\subsection{Results coronal plume}
Next we discuss the \emph{coronal plume model}. There is again a reflected wave visible.
In \cref{fig:plume-sections} we see that now there is a wave front with slightly higher density first, followed by a lower density wake by looking at the snapshots for $t=1.93e+3 s$ and $t=2.32e+3 s$.
We see that the the wave reflected when going from fast to slow magnetosonic speed is in phase, while the reflected wave going from low to high magnetosonic speed is out of phase.
This second fact was already observed in the case of the coronal hole, but can also be seen in the zoomed in images of the plume or the density profiles along a cut in the plume.
Between the frames at $t=2.1e+3 s$ and $t=2.3+e3$ in \cref{fig:plume-frames}  or $t=1.93e+3 s$ and $t=2.32e+3 s$ in \cref{fig:plume-sections}, the wave inside the plume reflects at the boundary and switches from a significantly higher density than the surrounding area to lower density, thus switching phase.

\begin{figure}[H]
	%\hspace{-1cm}
	\centering
	\includegraphics[width=\linewidth]{images/plume-frames.pdf}
	\caption{Plots of density profiles for the \emph{coronal plume model}. Same conventions used as in \cref{fig:hole-frames}}
	\label{fig:plume-frames}
\end{figure}

In this case, the transmitted wave is, as expected, slower than the wave outside the hole.
This is clearly visible as the line corresponding to $t=2.32e+3 s$ in the second plot of \cref{fig:plume-sections}.
The original wave front is clearly visible with in its wake a small spike. This is the transmitted wave.
Furthermore, the wave gets captured inside the plume, which functions as an echo chamber, emitting secondary wave fronts.
Such a secondary wave can be seen as the red ring forming in the last frame in $\cref{fig:plume-frames}$.

In the plume, the formation of a caustic is observed in the fourth frame of \cref{fig:plume-frames}. 
The density increase caused by the wave is concentrated in almost a single point here. 
This is due to the wave refracting when entering the hole which makes the plume function like a lens.
A not perfectly symmetrical shape of the plume will probably lead to less striking caustics.
In fact, there were also caustics in the coronal hole model.
In the frames at the bottom in figure \cref{fig:hole-frames} we see bright edges where the reflected/transmitted wave meets the main wave.

We see again that the structure is moved slightly along the direction of propagation of the wave. 
If there is a deformation of the coronal plume, it is a lot smaller than in the case of the coronal hole model.

Just as in the coronal hole model the wave is deformed by the interaction with the structure.
The transmitted wave forms a secondary wave front behind the main wave front, which is deformed by diffraction around the hole.

\begin{figure}[H]
	%\hspace{-1cm}
	\centering
	\includegraphics[width=\linewidth]{images/plume-sections.pdf}
	\caption{Density profiles along a cut through the center of the \emph{coronal plume} at different times, along the direction of propagation of the waves.
	The vertical dashed lines show the position of the wave at the bottom edge of the simulation domain.}
	\label{fig:plume-sections}
\end{figure}

