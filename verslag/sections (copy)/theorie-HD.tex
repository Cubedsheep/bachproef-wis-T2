While the main focus of this bachelor project is the numerical modeling of waves in the solar corona, some theoretical background is important to frame the results of our simulations.
Furthermore, this knowledge gives some insight in the assumptions that are made in deriving the magnetohydrodynamic (MHD) equations and when they are valid.

\subsection*{Hydrodynamics}

\subsection{Hydrodynamic fluid equations}
The theory in this section is adapted from \cite{notes-fluid-dynamics}. For the first task a non-viscous Newtonian fluid is considered. Heat conduction and dissipation is neglected as well.
This type of fluid obeys the Euler equations for conservation of mass, momentum and internal energy:

{\centering
\noindent \fbox{\parbox{0.5\linewidth}{
\begin{equation}
	\label{eq:ideal-fluid}
\begin{split}
	\frac{d\rho}{dt} + \rho \nabla \cdot \vec{v} &= 0\\
	\rho \frac{d\vec{v}}{dt} &= -\nabla p + \vec{F}\\
	\frac{dp}{dt} - \frac{\gamma p}{\rho} \frac{d\rho}{dt} &= 0
\end{split}
\end{equation}
}}\par}


These are the Euler equations in Lagrangian form, with time derivatives following the fluid, hence the total derivatives with respect to time.
PLUTO does the fluid simulation using a static grid, so we need the equations in Eulerian form with partial derivatives with respect to time.
This change of derivatives can be carried out using the following relation, found in \cite{notes-fluid-dynamics} in section 2.4:
\begin{equation}
	\frac{df}{dt} = \frac{\partial f}{\partial t} + (\vec{v} \cdot \nabla) f
	\label{eq:relation-total-partial}
\end{equation}
where $f(x,y,z,t)$ is a function that describes a property of the fluid. The equations can also be rederived using an Eulerian view. In any case the result is the same:

{\centering 
\noindent \fbox{\parbox{0.7\linewidth}{
\begin{align*}
	\frac{\partial \rho}{\partial t} + \nabla \cdot (\rho \vec{v}) &= 0\\
	\frac{\partial}{\partial t} (\rho \vec{v}) &= \nabla \cdot (-p - \rho \vec{v}\vec{v}) + \vec{F}\\
\frac{\partial}{\partial t} \left( \rho \left( \frac{v^2}{2} + \mathcal{U} \right)  \right)  &= \vec{F} \cdot \vec{v} - \nabla \cdot \left( \rho \left( \frac{v^2}{2}+\mathcal{U}\right)\vec{v} + p\vec{v})  \right) 
\end{align*}}}
\par}

Next introduce the variable $\vec{m}=\rho \vec{v}$, the momentum density. The energy density $\mathcal{U}$ can be split in the thermal energy $\rho e$ and gravitational potential energy $\rho\Phi$. Let $E_t = e\rho + \frac{v^2}{2}$ The only external force is $\vec{F} = \rho\vec{g}$
carrying out these substitutions leads to the equations in section 6 in the PLUTO manual \cite{pluto-manual}: \unsure{is the expression for $\vec{F}$ correct? Looks to be different in equation 2 and 3}

{\centering 
\noindent \fbox{\parbox{.7\linewidth}{
\begin{equation}
	\label{eq:ideal-fluid-pluto}
\begin{split}
	\frac{\partial \rho}{\partial t} + \nabla \cdot \vec{m} &= 0\\
	\frac{\partial \vec{m}}{\partial t} + \nabla \cdot \left( \vec{m}\vec{v} + p \right) &= -\rho\nabla\Phi + \rho \vec{g}\\
	\frac{\partial}{\partial t} \left( E_t+\rho\Phi \right)  + \nabla \cdot \left( \left( E_t + p + \rho\Phi \right) \vec{v} \right) &= \vec{m}\cdot\vec{g}
\end{split}
\end{equation}}}
\par}

Together with an equation of state $\rho e = \rho e(p, \rho)$, which gives the thermal energy as a function of $p$ and $\rho$.
In the remainder of this paper a calorically ideal gas is assumed. This is a gas for which the adiabatic constant $\gamma$ obeys:
\begin{equation}
	\gamma = \frac{f+2}{f}
	\label{eq:adiabatic constant}
\end{equation}
where $f$ is the number of degrees of freedom. the previous relation can be rewritten as
\begin{equation*}
	f = \frac{2}{\gamma-1}
\end{equation*}
And by substituting this equation in the equation that expresses the energy as a function of degrees of freedom the closure relation $\rho e= \rho e(\rho, p)$ is found:
\improvement{Add a reference for this energy equation}
\begin{equation*}
	E_t = \rho e = \frac{f}{2}nk_BT = \frac{p}{\gamma-1}
\end{equation*}
\todo[inline]{short discussion of assumptions made (no viscosity and heat conduction, callorically ideal gas)}
\todo[inline]{afleiding golven, groepssnelheid}
\subsection{Hydrodynamic linear waves}
We start again from the ideal fluid equations as given in \autoref{eq:ideal-fluid} and linearize them.
For this we rewrite the quantities $\rho$ and $p$ as a background density $\rho_0$ and pressure $p_0$ with slight deviations $\rho_1$, $p_1$.
Furthermore it is assumed that there are no external forces acting on the fluid. The Linearized equations are:

{\centering 
\noindent \fbox{\parbox{0.5\linewidth}{
\begin{equation}
	\label{eq:ideal-fluid-linear}
	\begin{split}
		\frac{\partial\rho_1}{\partial t} + \rho_0\nabla\cdot \vec{v} &= 0\\
		\rho_0 \frac{\partial \vec{v}}{\partial t} &= -\nabla p_1\\
		\frac{\partial p_1}{\partial t} &= \frac{\gamma p_0}{\rho_0} \frac{\partial\rho_1}{\partial t}
	\end{split}
\end{equation}
}}
\par}
By acting with $\nabla$ on the second equation and using the first to substitute $\rho_0 \nabla \cdot \vec{v}$ we find the following relation:
\begin{equation*}
	\frac{\partial^2 \rho_1}{\partial t^2} = -\nabla^2 p_1
\end{equation*}
Acting with $ \frac{\partial}{\partial t}$ on the last equation and substituting the previous expression yields
\begin{equation*}
	\frac{\partial^2 p_1}{\partial t^2} + \frac{\gamma p_0}{\rho_0} \nabla^2 p_1 = 0
\end{equation*}
which is the wave equation with $v_s = \sqrt{ \frac{\gamma p_0}{\rho_0} }$ the phase speed of the wave.
Similar expressions are found for the other variables. \improvement{Referentie?}
this wave speed can be found by substituting a plan wave of the form $p_1 = A \exp \left( i(\omega t - \vec{k}\cdot\vec{x}) \right) $. Substituting this expression in the wave equation for $p_1$ leads to the dispersion relation:
\begin{equation}
	\omega^2 = k^2v_s^2.
\end{equation}
The phase velocity is given by
\begin{equation}
	v_{ph} = \frac{\partial \omega}{\partial k} = v_s
\end{equation}
\improvement{reference for this relation for the phase speed?}
from which we conclude that these waves are non-dispersive.

\subsection{Hydrodynamic shocks}

Now we shall reconsider one of the least convincing assumptions made for the derivations of the fluid equations: that of perfectly continuous background variables. In reality, we might encounter very sudden changes in the scalar variable density $\rho$ and vectorial variable velocity $\vec{v}$. To have the theory of ideal fluids take this into account, we can introduce these jumps in the variables as mathematical discontinuities. This discontinuity is appropriately called a 'shock'. We are interested in how this shock moves through the fluid. The derivation of its motion is quite straight forward.\\
\\
Start from the continuity equation in its Eulerian form in 1D

\begin{equation}
\label{eq:continuity-Euler}
\frac{\partial \rho}{\partial t} + \frac{\partial(\rho v)}{\partial x} = 0
\end{equation}

Of course, this equation assumes that $\rho$ and $\rho v$ are continuous variables with continuous partial derivatives. Rewrite the equation so that over a distance $\Delta x$ and a duration $\Delta t$ the variables $\rho$ and $\rho v$ experience a change $\Delta\rho$ and $\Delta \rho v$. This gives the much less elegant version 

$$ \frac{\Delta \rho}{\Delta t} + \frac{\Delta(\rho v)}{\Delta x} = 0 \ . $$

If this were the perfectly continuous case we would now let $\Delta x, \Delta t \to 0$, resulting in \autoref{eq:continuity-Euler}. However, we might also say that the transition is not smooth and that for $\Delta x, \Delta t \to 0$ the jump remains: $ \Delta \rho, \Delta \rho v \to \Delta \rho, \Delta \rho v $. Rewrite the equations to see what this means:

$$ \frac{\Delta x}{\Delta t} \Delta \rho + \Delta(\rho v) . $$

Then for $\Delta x, \Delta t \to 0$ we get 

\begin{equation}
\label{eq:HD-shock-condition}
\frac{\partial x}{\partial t} \Delta \rho + \Delta(\rho v) = -V_S \Delta \rho + \Delta(\rho v) = 0
\end{equation}

where $V_S$ is the shock speed. This relation is the hydrodynamic shock condition. To generalize it beyond 1D, it suffices to take $\vec{v} \cdot \vec{n}$ instead of $v$ where $\vec{n}$ is the unit normal vector on the shock wave front pointing towards the region with lower pressure. It looks as follows

\begin{equation}
\label{eq:HD-3D-shock-condition}
- V_S \Delta(\rho v) + \Delta(\rho \ \vec{v} \cdot \vec{n}) = 0 \ .
\end{equation}

The minus sign in front of $V_S$ is merely a matter of orientation. In \autoref{eq:HD-shock-condition} the orientation is along the positive x-axis. In \autoref{eq:HD-3D-shock-condition} it is along the unit vector $\vec{n}$. This is the first of the three \textit{Rankine–Hugoniot} relations. The other two can analogously be derived from the Eulerian form of momentum and energy equations in \autoref{eq:ideal-fluid}. The three Rankine–Hugoniot conditions are

\begin{equation}
\label{HD-shock-conditions}
\begin{split}
V_S \Delta \rho &= \vec{n}\cdot\Delta(\rho \ \vec{v})\\
V_S \Delta (\rho \ \vec{v}) &= \vec{n} \cdot \Delta(\rho \ \vec{v} \ \vec{v} + p \ \I) \\
V_S \Delta E_t &= \vec{n} \cdot \Delta \big (\rho \ (e + \frac{v^2}{2} + \frac{p}{\rho}) \ \vec{v} \big)
\end{split}
\end{equation}







\todo[inline]{Derivation of MHD equation, discussion of the integration scheme used}
\todo[inline]{Test informatievaardigheden!!!}

